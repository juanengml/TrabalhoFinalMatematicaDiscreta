
% Default to the notebook output style

    


% Inherit from the specified cell style.




    
\documentclass[11pt]{article}

    
    
    \usepackage[T1]{fontenc}
    % Nicer default font (+ math font) than Computer Modern for most use cases
    \usepackage{mathpazo}

    % Basic figure setup, for now with no caption control since it's done
    % automatically by Pandoc (which extracts ![](path) syntax from Markdown).
    \usepackage{graphicx}
    % We will generate all images so they have a width \maxwidth. This means
    % that they will get their normal width if they fit onto the page, but
    % are scaled down if they would overflow the margins.
    \makeatletter
    \def\maxwidth{\ifdim\Gin@nat@width>\linewidth\linewidth
    \else\Gin@nat@width\fi}
    \makeatother
    \let\Oldincludegraphics\includegraphics
    % Set max figure width to be 80% of text width, for now hardcoded.
    \renewcommand{\includegraphics}[1]{\Oldincludegraphics[width=.8\maxwidth]{#1}}
    % Ensure that by default, figures have no caption (until we provide a
    % proper Figure object with a Caption API and a way to capture that
    % in the conversion process - todo).
    \usepackage{caption}
    \DeclareCaptionLabelFormat{nolabel}{}
    \captionsetup{labelformat=nolabel}

    \usepackage{adjustbox} % Used to constrain images to a maximum size 
    \usepackage{xcolor} % Allow colors to be defined
    \usepackage{enumerate} % Needed for markdown enumerations to work
    \usepackage{geometry} % Used to adjust the document margins
    \usepackage{amsmath} % Equations
    \usepackage{amssymb} % Equations
    \usepackage{textcomp} % defines textquotesingle
    % Hack from http://tex.stackexchange.com/a/47451/13684:
    \AtBeginDocument{%
        \def\PYZsq{\textquotesingle}% Upright quotes in Pygmentized code
    }
    \usepackage{upquote} % Upright quotes for verbatim code
    \usepackage{eurosym} % defines \euro
    \usepackage[mathletters]{ucs} % Extended unicode (utf-8) support
    \usepackage[utf8x]{inputenc} % Allow utf-8 characters in the tex document
    \usepackage{fancyvrb} % verbatim replacement that allows latex
    \usepackage{grffile} % extends the file name processing of package graphics 
                         % to support a larger range 
    % The hyperref package gives us a pdf with properly built
    % internal navigation ('pdf bookmarks' for the table of contents,
    % internal cross-reference links, web links for URLs, etc.)
    \usepackage{hyperref}
    \usepackage{longtable} % longtable support required by pandoc >1.10
    \usepackage{booktabs}  % table support for pandoc > 1.12.2
    \usepackage[inline]{enumitem} % IRkernel/repr support (it uses the enumerate* environment)
    \usepackage[normalem]{ulem} % ulem is needed to support strikethroughs (\sout)
                                % normalem makes italics be italics, not underlines
    

    
    
    % Colors for the hyperref package
    \definecolor{urlcolor}{rgb}{0,.145,.698}
    \definecolor{linkcolor}{rgb}{.71,0.21,0.01}
    \definecolor{citecolor}{rgb}{.12,.54,.11}

    % ANSI colors
    \definecolor{ansi-black}{HTML}{3E424D}
    \definecolor{ansi-black-intense}{HTML}{282C36}
    \definecolor{ansi-red}{HTML}{E75C58}
    \definecolor{ansi-red-intense}{HTML}{B22B31}
    \definecolor{ansi-green}{HTML}{00A250}
    \definecolor{ansi-green-intense}{HTML}{007427}
    \definecolor{ansi-yellow}{HTML}{DDB62B}
    \definecolor{ansi-yellow-intense}{HTML}{B27D12}
    \definecolor{ansi-blue}{HTML}{208FFB}
    \definecolor{ansi-blue-intense}{HTML}{0065CA}
    \definecolor{ansi-magenta}{HTML}{D160C4}
    \definecolor{ansi-magenta-intense}{HTML}{A03196}
    \definecolor{ansi-cyan}{HTML}{60C6C8}
    \definecolor{ansi-cyan-intense}{HTML}{258F8F}
    \definecolor{ansi-white}{HTML}{C5C1B4}
    \definecolor{ansi-white-intense}{HTML}{A1A6B2}

    % commands and environments needed by pandoc snippets
    % extracted from the output of `pandoc -s`
    \providecommand{\tightlist}{%
      \setlength{\itemsep}{0pt}\setlength{\parskip}{0pt}}
    \DefineVerbatimEnvironment{Highlighting}{Verbatim}{commandchars=\\\{\}}
    % Add ',fontsize=\small' for more characters per line
    \newenvironment{Shaded}{}{}
    \newcommand{\KeywordTok}[1]{\textcolor[rgb]{0.00,0.44,0.13}{\textbf{{#1}}}}
    \newcommand{\DataTypeTok}[1]{\textcolor[rgb]{0.56,0.13,0.00}{{#1}}}
    \newcommand{\DecValTok}[1]{\textcolor[rgb]{0.25,0.63,0.44}{{#1}}}
    \newcommand{\BaseNTok}[1]{\textcolor[rgb]{0.25,0.63,0.44}{{#1}}}
    \newcommand{\FloatTok}[1]{\textcolor[rgb]{0.25,0.63,0.44}{{#1}}}
    \newcommand{\CharTok}[1]{\textcolor[rgb]{0.25,0.44,0.63}{{#1}}}
    \newcommand{\StringTok}[1]{\textcolor[rgb]{0.25,0.44,0.63}{{#1}}}
    \newcommand{\CommentTok}[1]{\textcolor[rgb]{0.38,0.63,0.69}{\textit{{#1}}}}
    \newcommand{\OtherTok}[1]{\textcolor[rgb]{0.00,0.44,0.13}{{#1}}}
    \newcommand{\AlertTok}[1]{\textcolor[rgb]{1.00,0.00,0.00}{\textbf{{#1}}}}
    \newcommand{\FunctionTok}[1]{\textcolor[rgb]{0.02,0.16,0.49}{{#1}}}
    \newcommand{\RegionMarkerTok}[1]{{#1}}
    \newcommand{\ErrorTok}[1]{\textcolor[rgb]{1.00,0.00,0.00}{\textbf{{#1}}}}
    \newcommand{\NormalTok}[1]{{#1}}
    
    % Additional commands for more recent versions of Pandoc
    \newcommand{\ConstantTok}[1]{\textcolor[rgb]{0.53,0.00,0.00}{{#1}}}
    \newcommand{\SpecialCharTok}[1]{\textcolor[rgb]{0.25,0.44,0.63}{{#1}}}
    \newcommand{\VerbatimStringTok}[1]{\textcolor[rgb]{0.25,0.44,0.63}{{#1}}}
    \newcommand{\SpecialStringTok}[1]{\textcolor[rgb]{0.73,0.40,0.53}{{#1}}}
    \newcommand{\ImportTok}[1]{{#1}}
    \newcommand{\DocumentationTok}[1]{\textcolor[rgb]{0.73,0.13,0.13}{\textit{{#1}}}}
    \newcommand{\AnnotationTok}[1]{\textcolor[rgb]{0.38,0.63,0.69}{\textbf{\textit{{#1}}}}}
    \newcommand{\CommentVarTok}[1]{\textcolor[rgb]{0.38,0.63,0.69}{\textbf{\textit{{#1}}}}}
    \newcommand{\VariableTok}[1]{\textcolor[rgb]{0.10,0.09,0.49}{{#1}}}
    \newcommand{\ControlFlowTok}[1]{\textcolor[rgb]{0.00,0.44,0.13}{\textbf{{#1}}}}
    \newcommand{\OperatorTok}[1]{\textcolor[rgb]{0.40,0.40,0.40}{{#1}}}
    \newcommand{\BuiltInTok}[1]{{#1}}
    \newcommand{\ExtensionTok}[1]{{#1}}
    \newcommand{\PreprocessorTok}[1]{\textcolor[rgb]{0.74,0.48,0.00}{{#1}}}
    \newcommand{\AttributeTok}[1]{\textcolor[rgb]{0.49,0.56,0.16}{{#1}}}
    \newcommand{\InformationTok}[1]{\textcolor[rgb]{0.38,0.63,0.69}{\textbf{\textit{{#1}}}}}
    \newcommand{\WarningTok}[1]{\textcolor[rgb]{0.38,0.63,0.69}{\textbf{\textit{{#1}}}}}
    
    
    % Define a nice break command that doesn't care if a line doesn't already
    % exist.
    \def\br{\hspace*{\fill} \\* }
    % Math Jax compatability definitions
    \def\gt{>}
    \def\lt{<}
    % Document parameters
    \title{Trabalho Pr?tico de Matem?tica Discreta}
    
    
    

    % Pygments definitions
    
\makeatletter
\def\PY@reset{\let\PY@it=\relax \let\PY@bf=\relax%
    \let\PY@ul=\relax \let\PY@tc=\relax%
    \let\PY@bc=\relax \let\PY@ff=\relax}
\def\PY@tok#1{\csname PY@tok@#1\endcsname}
\def\PY@toks#1+{\ifx\relax#1\empty\else%
    \PY@tok{#1}\expandafter\PY@toks\fi}
\def\PY@do#1{\PY@bc{\PY@tc{\PY@ul{%
    \PY@it{\PY@bf{\PY@ff{#1}}}}}}}
\def\PY#1#2{\PY@reset\PY@toks#1+\relax+\PY@do{#2}}

\expandafter\def\csname PY@tok@gd\endcsname{\def\PY@tc##1{\textcolor[rgb]{0.63,0.00,0.00}{##1}}}
\expandafter\def\csname PY@tok@gu\endcsname{\let\PY@bf=\textbf\def\PY@tc##1{\textcolor[rgb]{0.50,0.00,0.50}{##1}}}
\expandafter\def\csname PY@tok@gt\endcsname{\def\PY@tc##1{\textcolor[rgb]{0.00,0.27,0.87}{##1}}}
\expandafter\def\csname PY@tok@gs\endcsname{\let\PY@bf=\textbf}
\expandafter\def\csname PY@tok@gr\endcsname{\def\PY@tc##1{\textcolor[rgb]{1.00,0.00,0.00}{##1}}}
\expandafter\def\csname PY@tok@cm\endcsname{\let\PY@it=\textit\def\PY@tc##1{\textcolor[rgb]{0.25,0.50,0.50}{##1}}}
\expandafter\def\csname PY@tok@vg\endcsname{\def\PY@tc##1{\textcolor[rgb]{0.10,0.09,0.49}{##1}}}
\expandafter\def\csname PY@tok@vi\endcsname{\def\PY@tc##1{\textcolor[rgb]{0.10,0.09,0.49}{##1}}}
\expandafter\def\csname PY@tok@vm\endcsname{\def\PY@tc##1{\textcolor[rgb]{0.10,0.09,0.49}{##1}}}
\expandafter\def\csname PY@tok@mh\endcsname{\def\PY@tc##1{\textcolor[rgb]{0.40,0.40,0.40}{##1}}}
\expandafter\def\csname PY@tok@cs\endcsname{\let\PY@it=\textit\def\PY@tc##1{\textcolor[rgb]{0.25,0.50,0.50}{##1}}}
\expandafter\def\csname PY@tok@ge\endcsname{\let\PY@it=\textit}
\expandafter\def\csname PY@tok@vc\endcsname{\def\PY@tc##1{\textcolor[rgb]{0.10,0.09,0.49}{##1}}}
\expandafter\def\csname PY@tok@il\endcsname{\def\PY@tc##1{\textcolor[rgb]{0.40,0.40,0.40}{##1}}}
\expandafter\def\csname PY@tok@go\endcsname{\def\PY@tc##1{\textcolor[rgb]{0.53,0.53,0.53}{##1}}}
\expandafter\def\csname PY@tok@cp\endcsname{\def\PY@tc##1{\textcolor[rgb]{0.74,0.48,0.00}{##1}}}
\expandafter\def\csname PY@tok@gi\endcsname{\def\PY@tc##1{\textcolor[rgb]{0.00,0.63,0.00}{##1}}}
\expandafter\def\csname PY@tok@gh\endcsname{\let\PY@bf=\textbf\def\PY@tc##1{\textcolor[rgb]{0.00,0.00,0.50}{##1}}}
\expandafter\def\csname PY@tok@ni\endcsname{\let\PY@bf=\textbf\def\PY@tc##1{\textcolor[rgb]{0.60,0.60,0.60}{##1}}}
\expandafter\def\csname PY@tok@nl\endcsname{\def\PY@tc##1{\textcolor[rgb]{0.63,0.63,0.00}{##1}}}
\expandafter\def\csname PY@tok@nn\endcsname{\let\PY@bf=\textbf\def\PY@tc##1{\textcolor[rgb]{0.00,0.00,1.00}{##1}}}
\expandafter\def\csname PY@tok@no\endcsname{\def\PY@tc##1{\textcolor[rgb]{0.53,0.00,0.00}{##1}}}
\expandafter\def\csname PY@tok@na\endcsname{\def\PY@tc##1{\textcolor[rgb]{0.49,0.56,0.16}{##1}}}
\expandafter\def\csname PY@tok@nb\endcsname{\def\PY@tc##1{\textcolor[rgb]{0.00,0.50,0.00}{##1}}}
\expandafter\def\csname PY@tok@nc\endcsname{\let\PY@bf=\textbf\def\PY@tc##1{\textcolor[rgb]{0.00,0.00,1.00}{##1}}}
\expandafter\def\csname PY@tok@nd\endcsname{\def\PY@tc##1{\textcolor[rgb]{0.67,0.13,1.00}{##1}}}
\expandafter\def\csname PY@tok@ne\endcsname{\let\PY@bf=\textbf\def\PY@tc##1{\textcolor[rgb]{0.82,0.25,0.23}{##1}}}
\expandafter\def\csname PY@tok@nf\endcsname{\def\PY@tc##1{\textcolor[rgb]{0.00,0.00,1.00}{##1}}}
\expandafter\def\csname PY@tok@si\endcsname{\let\PY@bf=\textbf\def\PY@tc##1{\textcolor[rgb]{0.73,0.40,0.53}{##1}}}
\expandafter\def\csname PY@tok@s2\endcsname{\def\PY@tc##1{\textcolor[rgb]{0.73,0.13,0.13}{##1}}}
\expandafter\def\csname PY@tok@nt\endcsname{\let\PY@bf=\textbf\def\PY@tc##1{\textcolor[rgb]{0.00,0.50,0.00}{##1}}}
\expandafter\def\csname PY@tok@nv\endcsname{\def\PY@tc##1{\textcolor[rgb]{0.10,0.09,0.49}{##1}}}
\expandafter\def\csname PY@tok@s1\endcsname{\def\PY@tc##1{\textcolor[rgb]{0.73,0.13,0.13}{##1}}}
\expandafter\def\csname PY@tok@dl\endcsname{\def\PY@tc##1{\textcolor[rgb]{0.73,0.13,0.13}{##1}}}
\expandafter\def\csname PY@tok@ch\endcsname{\let\PY@it=\textit\def\PY@tc##1{\textcolor[rgb]{0.25,0.50,0.50}{##1}}}
\expandafter\def\csname PY@tok@m\endcsname{\def\PY@tc##1{\textcolor[rgb]{0.40,0.40,0.40}{##1}}}
\expandafter\def\csname PY@tok@gp\endcsname{\let\PY@bf=\textbf\def\PY@tc##1{\textcolor[rgb]{0.00,0.00,0.50}{##1}}}
\expandafter\def\csname PY@tok@sh\endcsname{\def\PY@tc##1{\textcolor[rgb]{0.73,0.13,0.13}{##1}}}
\expandafter\def\csname PY@tok@ow\endcsname{\let\PY@bf=\textbf\def\PY@tc##1{\textcolor[rgb]{0.67,0.13,1.00}{##1}}}
\expandafter\def\csname PY@tok@sx\endcsname{\def\PY@tc##1{\textcolor[rgb]{0.00,0.50,0.00}{##1}}}
\expandafter\def\csname PY@tok@bp\endcsname{\def\PY@tc##1{\textcolor[rgb]{0.00,0.50,0.00}{##1}}}
\expandafter\def\csname PY@tok@c1\endcsname{\let\PY@it=\textit\def\PY@tc##1{\textcolor[rgb]{0.25,0.50,0.50}{##1}}}
\expandafter\def\csname PY@tok@fm\endcsname{\def\PY@tc##1{\textcolor[rgb]{0.00,0.00,1.00}{##1}}}
\expandafter\def\csname PY@tok@o\endcsname{\def\PY@tc##1{\textcolor[rgb]{0.40,0.40,0.40}{##1}}}
\expandafter\def\csname PY@tok@kc\endcsname{\let\PY@bf=\textbf\def\PY@tc##1{\textcolor[rgb]{0.00,0.50,0.00}{##1}}}
\expandafter\def\csname PY@tok@c\endcsname{\let\PY@it=\textit\def\PY@tc##1{\textcolor[rgb]{0.25,0.50,0.50}{##1}}}
\expandafter\def\csname PY@tok@mf\endcsname{\def\PY@tc##1{\textcolor[rgb]{0.40,0.40,0.40}{##1}}}
\expandafter\def\csname PY@tok@err\endcsname{\def\PY@bc##1{\setlength{\fboxsep}{0pt}\fcolorbox[rgb]{1.00,0.00,0.00}{1,1,1}{\strut ##1}}}
\expandafter\def\csname PY@tok@mb\endcsname{\def\PY@tc##1{\textcolor[rgb]{0.40,0.40,0.40}{##1}}}
\expandafter\def\csname PY@tok@ss\endcsname{\def\PY@tc##1{\textcolor[rgb]{0.10,0.09,0.49}{##1}}}
\expandafter\def\csname PY@tok@sr\endcsname{\def\PY@tc##1{\textcolor[rgb]{0.73,0.40,0.53}{##1}}}
\expandafter\def\csname PY@tok@mo\endcsname{\def\PY@tc##1{\textcolor[rgb]{0.40,0.40,0.40}{##1}}}
\expandafter\def\csname PY@tok@kd\endcsname{\let\PY@bf=\textbf\def\PY@tc##1{\textcolor[rgb]{0.00,0.50,0.00}{##1}}}
\expandafter\def\csname PY@tok@mi\endcsname{\def\PY@tc##1{\textcolor[rgb]{0.40,0.40,0.40}{##1}}}
\expandafter\def\csname PY@tok@kn\endcsname{\let\PY@bf=\textbf\def\PY@tc##1{\textcolor[rgb]{0.00,0.50,0.00}{##1}}}
\expandafter\def\csname PY@tok@cpf\endcsname{\let\PY@it=\textit\def\PY@tc##1{\textcolor[rgb]{0.25,0.50,0.50}{##1}}}
\expandafter\def\csname PY@tok@kr\endcsname{\let\PY@bf=\textbf\def\PY@tc##1{\textcolor[rgb]{0.00,0.50,0.00}{##1}}}
\expandafter\def\csname PY@tok@s\endcsname{\def\PY@tc##1{\textcolor[rgb]{0.73,0.13,0.13}{##1}}}
\expandafter\def\csname PY@tok@kp\endcsname{\def\PY@tc##1{\textcolor[rgb]{0.00,0.50,0.00}{##1}}}
\expandafter\def\csname PY@tok@w\endcsname{\def\PY@tc##1{\textcolor[rgb]{0.73,0.73,0.73}{##1}}}
\expandafter\def\csname PY@tok@kt\endcsname{\def\PY@tc##1{\textcolor[rgb]{0.69,0.00,0.25}{##1}}}
\expandafter\def\csname PY@tok@sc\endcsname{\def\PY@tc##1{\textcolor[rgb]{0.73,0.13,0.13}{##1}}}
\expandafter\def\csname PY@tok@sb\endcsname{\def\PY@tc##1{\textcolor[rgb]{0.73,0.13,0.13}{##1}}}
\expandafter\def\csname PY@tok@sa\endcsname{\def\PY@tc##1{\textcolor[rgb]{0.73,0.13,0.13}{##1}}}
\expandafter\def\csname PY@tok@k\endcsname{\let\PY@bf=\textbf\def\PY@tc##1{\textcolor[rgb]{0.00,0.50,0.00}{##1}}}
\expandafter\def\csname PY@tok@se\endcsname{\let\PY@bf=\textbf\def\PY@tc##1{\textcolor[rgb]{0.73,0.40,0.13}{##1}}}
\expandafter\def\csname PY@tok@sd\endcsname{\let\PY@it=\textit\def\PY@tc##1{\textcolor[rgb]{0.73,0.13,0.13}{##1}}}

\def\PYZbs{\char`\\}
\def\PYZus{\char`\_}
\def\PYZob{\char`\{}
\def\PYZcb{\char`\}}
\def\PYZca{\char`\^}
\def\PYZam{\char`\&}
\def\PYZlt{\char`\<}
\def\PYZgt{\char`\>}
\def\PYZsh{\char`\#}
\def\PYZpc{\char`\%}
\def\PYZdl{\char`\$}
\def\PYZhy{\char`\-}
\def\PYZsq{\char`\'}
\def\PYZdq{\char`\"}
\def\PYZti{\char`\~}
% for compatibility with earlier versions
\def\PYZat{@}
\def\PYZlb{[}
\def\PYZrb{]}
\makeatother


    % Exact colors from NB
    \definecolor{incolor}{rgb}{0.0, 0.0, 0.5}
    \definecolor{outcolor}{rgb}{0.545, 0.0, 0.0}



    
    % Prevent overflowing lines due to hard-to-break entities
    \sloppy 
    % Setup hyperref package
    \hypersetup{
      breaklinks=true,  % so long urls are correctly broken across lines
      colorlinks=true,
      urlcolor=urlcolor,
      linkcolor=linkcolor,
      citecolor=citecolor,
      }
    % Slightly bigger margins than the latex defaults
    
    \geometry{verbose,tmargin=1in,bmargin=1in,lmargin=1in,rmargin=1in}
    
    

    \begin{document}
    
    
    \maketitle
    
    

    
    Trabalho Prático de Matemática Discreta: Análise de Algoritmos de
Ordenação

    \begin{longtable}[]{@{}ll@{}}
\toprule
Membros & Nome\tabularnewline
\midrule
\endhead
01 & Juan Manoel\tabularnewline
02 & Robson Barreto\tabularnewline
03 & Yang Jin Samara cavalari\tabularnewline
04 & Gustavo Jeromine\tabularnewline
05 & Vladimir da Silva Borges\tabularnewline
\bottomrule
\end{longtable}

    \subsubsection{1 - Introdução}\label{introduuxe7uxe3o}

Deve conter:(Vladimir e Robson) * Uma breve descrição do objetivo do
trabalho * O que são algoritmos de ordenação, o que fazem, como
funcionam, onde podem ser aplicados, importância etc * Uma breve
introdução aos algoritmos escolhidos

    Objetivo:

Escrever uma análise descrevendo e comparando dois algoritmos de
ordenação distintos.

Objetivos Específicos:

\begin{itemize}
\item
  Buscar por dois algoritmos distintos de ordenação na literatura.
\item
  Descrever como funcionam, de onde surgiram, etc.
\item
  Buscar na literatura pela complexidade assintótica no melhor e pior
  caso em notação Ozão ou Theta em relação ao tempo de execução/número
  de operações primitivas realizadas.
\item
  Analizar as diferenças de complexidade entre os dois algoritmos e
  discutir a respeito da eficiência dos mesmos.
\end{itemize}

    \begin{Verbatim}[commandchars=\\\{\}]
{\color{incolor}In [{\color{incolor}1}]:} \PY{k+kn}{from} \PY{n+nn}{IPython.display} \PY{k+kn}{import} \PY{n}{HTML}
        \PY{k+kn}{import} \PY{n+nn}{time}
\end{Verbatim}


    \subsubsection{2 - Descrição e Análise do Algoritmo
1}\label{descriuxe7uxe3o-e-anuxe1lise-do-algoritmo-1}

Deve conter: * Descrição completa do primeiro algoritmo escolhido: nome,
origem, estratégia usada, e como funciona * Estrutura do algoritmo em
pseudocódigo * Citações para a descrição do algoritmo e pseudocódigo

    \subparagraph{Descrição completa do primeiro algoritmo escolhido: nome,
origem, estratégia usada, e como
funciona}\label{descriuxe7uxe3o-completa-do-primeiro-algoritmo-escolhido-nome-origem-estratuxe9gia-usada-e-como-funciona}

 Nome: Merge Sort Origem: Existem evidências de que o algoritmo foi
proposto por John Von Neumann em 1945. Essa discussão existe, por que ao
estudar as várias contribuições que ele fez é, ao mesmo tempo, complexa
e fascinante. Essa complexidade devesse em parte a existência de muitas
fontes de informação, algumas pouco é dificilmente acessíveis, outras
discordantes entre si ou polêmicas. Outras contruibuições atribuem ao
Knuth, que argumentou no seu livro `Arte de Programação Computacional:
Ordenando e Procurando' que Von Neumann foi o primeiro a descrever a
idéia. Estratégia usada: tenicas de classificação - Ordenação por
partição(dividir e conquistar) o mergesort é classificado como ordenação
por partição, que parte do princípio de "dividir para conquistar". Este
princípio é uma técnica que foi utilizada pela primeira vez por Anatolii
Karatsuba em 1960 e consiste em dividir um problema maior em problemas
pequenos, e sucessivamente até que o mesmo seja resolvido diretamente.

 Como funciona: a técnica realiza-se em três fases.

\begin{enumerate}
\def\labelenumi{\arabic{enumi})}
\item
  Divisão: o problema maior é dividido em problemas menores e os
  problemas menores obtidos são novamente divididos sucessivamente de
  maneira recursiva.
\item
  Conquista: o resultado do problema é calculado quando o problema é
  pequeno o suficiente.
\item
  Combinação: os resultados dos problemas menores são combinados até que
  seja obtida a solução do problema maior. Algoritmos que utilizam o
  método de partição são caracterizados por serem os mais rápidos dentre
  os outros algoritmos pelo fato de sua complexidade ser, na maioria das
  situações, O(nlogn). Os dois representantes mais ilustres desta classe
  são o quicksort e o mergesort
\end{enumerate}

    \begin{Verbatim}[commandchars=\\\{\}]
{\color{incolor}In [{\color{incolor}2}]:} \PY{k}{print} \PY{l+s+s2}{\PYZdq{}}\PY{l+s+s2}{Aplicação Merge Sort}\PY{l+s+s2}{\PYZdq{}}
        \PY{n}{HTML}\PY{p}{(}\PY{l+s+s1}{\PYZsq{}}\PY{l+s+s1}{\PYZlt{}img src=}\PY{l+s+s1}{\PYZdq{}}\PY{l+s+s1}{./merge.gif}\PY{l+s+s1}{\PYZdq{}}\PY{l+s+s1}{ style=}\PY{l+s+s1}{\PYZdq{}}\PY{l+s+s1}{width:200px;height:200px;}\PY{l+s+s1}{\PYZdq{}}\PY{l+s+s1}{ \PYZgt{}}\PY{l+s+s1}{\PYZsq{}}\PY{p}{)}
\end{Verbatim}


    \begin{Verbatim}[commandchars=\\\{\}]
Aplicação Merge Sort

    \end{Verbatim}

\begin{Verbatim}[commandchars=\\\{\}]
{\color{outcolor}Out[{\color{outcolor}2}]:} <IPython.core.display.HTML object>
\end{Verbatim}
            
    \section{Estrutura do algoritmo em
pseudocódigo}\label{estrutura-do-algoritmo-em-pseudocuxf3digo}

    \\
MERGE-SORT(A, p, r)

\begin{verbatim}
if p < r then  
    q = ((p + r) / 2) //calcula o meio
    Merge-Sort(A, p, q)
    Merge-Sort(A, q + 1, r)
    Merge(A, p, q, r)
\end{verbatim}

Merge(A, p, q, r)

\begin{verbatim}
n1 = q - p + 1
n2 = r - q
sejam L[1 ... n1 + 1] e R[1 ... n2 + 1]
for i = 1 to n1
    L[i] = A[p + i - 1]
for j = 1 to n2
    R[j] = A[q + j]

i = 1
j = 1

for k = p to r
    if L[i] <= R[j] then A[k] = L[i]
        i = i + 1
    else A[k] = R[j]
        j = j + 1
        
        
\end{verbatim}

    \subsubsection{Citações para a descrição do algoritmo e
pseudocódigo}\label{citauxe7uxf5es-para-a-descriuxe7uxe3o-do-algoritmo-e-pseudocuxf3digo}
mergesort(A[0...n - 1], inicio, fim)
|   se(inicio < fim)
|   |   meio ← (inicio + fim) / 2 //calcula o meio
|   |   mergesort(A, inicio, meio) //ordena o subvetor esquerdo
|   |   mergesort(A, meio + 1, fim) //ordena o subvetor direito
|   |   merge(A, inicio, meio, fim) //funde os subvetores esquerdo e direito
|   fim_se
fim_mergesort

merge(A[0...n - 1], inicio, meio, fim)
|   tamEsq ← meio - inicio + 1 //tamanho do subvetor esquerdo
|   tamDir ← fim - meio //tamanho do subvetor direito
|   inicializar vetor Esq[0...tamEsq - 1]
|   inicializar vetor Dir[0...tamDir - 1]
|   para i ← 0 até tamEsq - 1
|   |   Esq[i] ← A[inicio + i] //elementos do subvetor esquerdo
|   fim_para
|   para j ← 0 até tamDir - 1
|   |   Dir[j] ← A[meio + 1 + j] //elementos do subvetor direito
|   fim_para
|   idxEsq ← 0 //índice do subvetor auxiliar esquerdo
|   idxDir ← 0 //índice do subvetor auxiliar direito
|   para k ← inicio até fim
|   |   se(idxEsq < tamEsq)
|   |   |   se(idxDir < tamDir)
|   |   |   |   se(Esq[idxEsq] < Dir[idxDir])
|   |   |   |   |   A[k] ← Esq[idxEsq]
|   |   |   |   |   idxEsq ← idxEsq + 1
|   |   |   |   senão
|   |   |   |   |   A[k] ← Dir[idxDir]
|   |   |   |   |   idxDir ← idxDir + 1
|   |   |   |   fim_se
|   |   |   senão
|   |   |   |   A[k] ← Esq[idxEsq]
|   |   |   |   idxEsq ← idxEsq + 1
|   |   |   fim_se
|   |   senão
|   |   |   A[k] ← Dir[idxDir]
|   |   |   idxDir ← idxDir + 1
|   |   fim_se
|   fim_para
fim_merge
    " Observe que o método merge utiliza dois vetores auxiliares. A
utilização desses vetores faz com o Merge Sort tenha complexidade O(n)
no espaço.

Por causa da cópia de elementos entre os vetores auxiliares e o vetor A,
a complexidade no tempo do método merge é Θ(n) ou O(n).

Alternativamente, podemos utilizar um único vetor auxiliar na ordenação,
porém a complexidade no tempo e no espaço será a mesma. "

{[}1{]} CORMEN, T. H. et al. Algoritmos: teoria e prática. 3 ed. Rio de
Janeiro: Elsevier, 2012.

\begin{center}\rule{0.5\linewidth}{\linethickness}\end{center}

    \paragraph{Codigo python}\label{codigo-python}

    \begin{Verbatim}[commandchars=\\\{\}]
{\color{incolor}In [{\color{incolor}3}]:} \PY{k}{def} \PY{n+nf}{merge}\PY{p}{(}\PY{n}{llist}\PY{p}{,} \PY{n}{rlist}\PY{p}{)}\PY{p}{:}
                \PY{n}{final} \PY{o}{=} \PY{p}{[}\PY{p}{]}
                \PY{k}{while} \PY{n}{llist} \PY{o+ow}{or} \PY{n}{rlist}\PY{p}{:}
                        \PY{k}{if} \PY{n+nb}{len}\PY{p}{(}\PY{n}{llist}\PY{p}{)} \PY{o+ow}{and} \PY{n+nb}{len}\PY{p}{(}\PY{n}{rlist}\PY{p}{)}\PY{p}{:}
                                \PY{k}{if} \PY{n}{llist}\PY{p}{[}\PY{l+m+mi}{0}\PY{p}{]} \PY{o}{\PYZlt{}} \PY{n}{rlist}\PY{p}{[}\PY{l+m+mi}{0}\PY{p}{]}\PY{p}{:}
                                        \PY{n}{final}\PY{o}{.}\PY{n}{append}\PY{p}{(}\PY{n}{llist}\PY{o}{.}\PY{n}{pop}\PY{p}{(}\PY{l+m+mi}{0}\PY{p}{)}\PY{p}{)}
                                        
                                \PY{k}{else}\PY{p}{:}
                                        \PY{n}{final}\PY{o}{.}\PY{n}{append}\PY{p}{(}\PY{n}{rlist}\PY{o}{.}\PY{n}{pop}\PY{p}{(}\PY{l+m+mi}{0}\PY{p}{)}\PY{p}{)}
                                         
        
                        \PY{k}{if} \PY{o+ow}{not} \PY{n+nb}{len}\PY{p}{(}\PY{n}{llist}\PY{p}{)}\PY{p}{:}
                                        \PY{k}{if} \PY{n+nb}{len}\PY{p}{(}\PY{n}{rlist}\PY{p}{)}\PY{p}{:}
                                              \PY{n}{final}\PY{o}{.}\PY{n}{append}\PY{p}{(}\PY{n}{rlist}\PY{o}{.}\PY{n}{pop}\PY{p}{(}\PY{l+m+mi}{0}\PY{p}{)}\PY{p}{)}
                                               
        
                        \PY{k}{if} \PY{o+ow}{not} \PY{n+nb}{len}\PY{p}{(}\PY{n}{rlist}\PY{p}{)}\PY{p}{:}
                                        \PY{k}{if} \PY{n+nb}{len}\PY{p}{(}\PY{n}{llist}\PY{p}{)}\PY{p}{:} 
                                              \PY{n}{final}\PY{o}{.}\PY{n}{append}\PY{p}{(}\PY{n}{llist}\PY{o}{.}\PY{n}{pop}\PY{p}{(}\PY{l+m+mi}{0}\PY{p}{)}\PY{p}{)}
        
                \PY{k}{return} \PY{n}{final}
        
        \PY{k}{def} \PY{n+nf}{merge\PYZus{}sort}\PY{p}{(}\PY{n+nb}{list}\PY{p}{)}\PY{p}{:}
                \PY{k}{if} \PY{n+nb}{len}\PY{p}{(}\PY{n+nb}{list}\PY{p}{)} \PY{o}{\PYZlt{}} \PY{l+m+mi}{2}\PY{p}{:} \PY{k}{return} \PY{n+nb}{list}
                \PY{n}{mid} \PY{o}{=} \PY{n+nb}{len}\PY{p}{(}\PY{n+nb}{list}\PY{p}{)} \PY{o}{/} \PY{l+m+mi}{2}
                \PY{k}{return} \PY{n}{merge}\PY{p}{(}\PY{n}{merge\PYZus{}sort}\PY{p}{(}\PY{n+nb}{list}\PY{p}{[}\PY{p}{:}\PY{n}{mid}\PY{p}{]}\PY{p}{)}\PY{p}{,} \PY{n}{merge\PYZus{}sort}\PY{p}{(}\PY{n+nb}{list}\PY{p}{[}\PY{n}{mid}\PY{p}{:}\PY{p}{]}\PY{p}{)}\PY{p}{)}
\end{Verbatim}


    \subsubsection{3 - Descrição e Análise do Algoritmo
2}\label{descriuxe7uxe3o-e-anuxe1lise-do-algoritmo-2}

Deve conter: * Descrição completa do segundo algoritmo escolhido: nome,
origem, estratégia usada, e como funciona * Estrutura do algoritmo em
pseudocódigo * Citações para a descrição do algoritmo e pseudocódigo

 Descrição completa do segundo algoritmo escolhido: nome, origem,
estratégia usada, e como funciona

Nome: Bubble Sort Origem: Origem Não Achei Estrategia Usada: é o
algoritmo de ordenação mais simples que funciona trocando repetidamente
os elementos adjacentes se eles estiverem na ordem errada. Como
funciona: Percorre o vetor inteiro comparando elementos adjacentes (dois
a dois) a estrategia de trocar as posições dos elementos se eles
estiverem fora de ordem o repita os dois passos acima com os primeiros
n-1 itens, depois com os primeiros n-2 itens, até que reste apenas um
item 

    \begin{Verbatim}[commandchars=\\\{\}]
{\color{incolor}In [{\color{incolor}4}]:} \PY{k}{print} \PY{l+s+s2}{\PYZdq{}}\PY{l+s+s2}{Aplicação Bubble Sort}\PY{l+s+s2}{\PYZdq{}}
        \PY{n}{HTML}\PY{p}{(}\PY{l+s+s1}{\PYZsq{}}\PY{l+s+s1}{\PYZlt{}img src=}\PY{l+s+s1}{\PYZdq{}}\PY{l+s+s1}{./Bubble.gif}\PY{l+s+s1}{\PYZdq{}}\PY{l+s+s1}{ style=}\PY{l+s+s1}{\PYZdq{}}\PY{l+s+s1}{width:200px;height:200px;}\PY{l+s+s1}{\PYZdq{}}\PY{l+s+s1}{ \PYZgt{}}\PY{l+s+s1}{\PYZsq{}}\PY{p}{)}
\end{Verbatim}


    \begin{Verbatim}[commandchars=\\\{\}]
Aplicação Bubble Sort

    \end{Verbatim}

\begin{Verbatim}[commandchars=\\\{\}]
{\color{outcolor}Out[{\color{outcolor}4}]:} <IPython.core.display.HTML object>
\end{Verbatim}
            
    \subsubsection{Estrutura do algoritmo em
pseudocódigo}\label{estrutura-do-algoritmo-em-pseudocuxf3digo}
procedure bubbleSort( A : lista de itens ordenaveis ) defined as:
  do
    trocado := false
    for each i in 0 to length( A ) - 2 do:
      // verificar se os elementos estão na ordem certa
      if A[ i ] > A[ i + 1 ] then
        // trocar elementos de lugar
        trocar( A[ i ], A[ i + 1 ] )
        trocado := true
      end if
    end for
  // enquanto houver elementos sendo reordenados.
  while trocado
end procedure
    \begin{center}\rule{0.5\linewidth}{\linethickness}\end{center}

\paragraph{Codigo Python}\label{codigo-python}

    \begin{Verbatim}[commandchars=\\\{\}]
{\color{incolor}In [{\color{incolor}5}]:} \PY{k}{def} \PY{n+nf}{bubble\PYZus{}sort}\PY{p}{(}\PY{n}{lista}\PY{p}{)}\PY{p}{:}
           \PY{n}{elementos} \PY{o}{=} \PY{n+nb}{len}\PY{p}{(}\PY{n}{lista}\PY{p}{)}\PY{o}{\PYZhy{}}\PY{l+m+mi}{1}
           \PY{n}{ordenado} \PY{o}{=} \PY{n+nb+bp}{False}
           \PY{k}{while} \PY{o+ow}{not} \PY{n}{ordenado}\PY{p}{:}
                \PY{n}{ordenado} \PY{o}{=} \PY{n+nb+bp}{True}
                \PY{k}{for} \PY{n}{i} \PY{o+ow}{in} \PY{n+nb}{range}\PY{p}{(}\PY{n}{elementos}\PY{p}{)}\PY{p}{:}
                     \PY{k}{if} \PY{n}{lista}\PY{p}{[}\PY{n}{i}\PY{p}{]} \PY{o}{\PYZgt{}} \PY{n}{lista}\PY{p}{[}\PY{n}{i}\PY{o}{+}\PY{l+m+mi}{1}\PY{p}{]}\PY{p}{:}
                         \PY{n}{lista}\PY{p}{[}\PY{n}{i}\PY{p}{]}\PY{p}{,} \PY{n}{lista}\PY{p}{[}\PY{n}{i}\PY{o}{+}\PY{l+m+mi}{1}\PY{p}{]} \PY{o}{=} \PY{n}{lista}\PY{p}{[}\PY{n}{i}\PY{o}{+}\PY{l+m+mi}{1}\PY{p}{]}\PY{p}{,}\PY{n}{lista}\PY{p}{[}\PY{n}{i}\PY{p}{]}
                         \PY{n}{ordenado} \PY{o}{=} \PY{n+nb+bp}{False}        
                         \PY{c+c1}{\PYZsh{}print(lista)}
           \PY{k}{return} \PY{n}{lista}
\end{Verbatim}


    \subsubsection{4 - Análise Assintótica
Comparativa}\label{anuxe1lise-assintuxf3tica-comparativa}

Deve conter: * Representação em Ozão ou Theta da complexidade do
algoritmo 1 no melhor e pior caso em relação ao tempo de execução *
Representação em Ozão ou Theta da complexidade do algoritmo 2 no melhor
e pior caso em relação ao tempo de execução * Discussão a respeito de
qual algoritmo é o mais eficiente

    Complexidade Algoritmo 1 Merge

\begin{longtable}[]{@{}lll@{}}
\toprule
Caso & Analise Assintótica & Tempo Execução\tabularnewline
\midrule
\endhead
Melhor & O(n log base 2 n) & 0.000979900360107\tabularnewline
Pior & O(n log base 2 n) & 0.0190689563751\tabularnewline
\bottomrule
\end{longtable}

    \begin{Verbatim}[commandchars=\\\{\}]
{\color{incolor}In [{\color{incolor}26}]:} \PY{n}{m} \PY{o}{=} \PY{n}{merge\PYZus{}sort}\PY{p}{(}\PY{n}{Metrica\PYZus{}merge}\PY{p}{)}
         \PY{k}{print} \PY{l+s+s2}{\PYZdq{}}\PY{l+s+s2}{Melhor Caso: }\PY{l+s+s2}{\PYZdq{}}\PY{p}{,}\PY{n}{m}\PY{p}{[}\PY{l+m+mi}{0}\PY{p}{]}
         \PY{k}{print} \PY{l+s+s2}{\PYZdq{}}\PY{l+s+s2}{Pior Caso: }\PY{l+s+s2}{\PYZdq{}}\PY{p}{,}\PY{n}{m}\PY{p}{[}\PY{o}{\PYZhy{}}\PY{l+m+mi}{1}\PY{p}{]}
\end{Verbatim}


    \begin{Verbatim}[commandchars=\\\{\}]
Melhor Caso:  0.000979900360107
Pior Caso:  0.0190689563751

    \end{Verbatim}

    Complexidade Algoritmo 2 Bubble

\begin{longtable}[]{@{}lll@{}}
\toprule
Caso & Analise Assintótica & Tempo de Execução\tabularnewline
\midrule
\endhead
Melhor & O(n²) & 0.000216007232666\tabularnewline
Pior & O(n²) & 0.00135803222656\tabularnewline
\bottomrule
\end{longtable}

    \begin{Verbatim}[commandchars=\\\{\}]
{\color{incolor}In [{\color{incolor}27}]:} \PY{n}{m} \PY{o}{=} \PY{n}{bubble\PYZus{}sort}\PY{p}{(}\PY{n}{Metrica\PYZus{}bubble}\PY{p}{)}
         \PY{k}{print} \PY{l+s+s2}{\PYZdq{}}\PY{l+s+s2}{Melhor Caso: }\PY{l+s+s2}{\PYZdq{}}\PY{p}{,}\PY{n}{m}\PY{p}{[}\PY{l+m+mi}{0}\PY{p}{]}
         \PY{k}{print} \PY{l+s+s2}{\PYZdq{}}\PY{l+s+s2}{Pior Caso: }\PY{l+s+s2}{\PYZdq{}}\PY{p}{,}\PY{n}{m}\PY{p}{[}\PY{o}{\PYZhy{}}\PY{l+m+mi}{1}\PY{p}{]}
\end{Verbatim}


    \begin{Verbatim}[commandchars=\\\{\}]
Melhor Caso:  0.000216007232666
Pior Caso:  0.00135803222656

    \end{Verbatim}

    \subsubsection{Discussão a respeito de qual algoritmo é o mais
eficiente}\label{discussuxe3o-a-respeito-de-qual-algoritmo-uxe9-o-mais-eficiente}

    Bubble sort por ser uma função quadrática começa a rodar mais
rapidamente enquanto o merge sort é uma função log-linear na base 2 e
demora mais, ou seja é mais lento. Enquanto a sua estabilidade ao longo
do tempo o merge sort é mais recomendado pois é estável.

    O Bubble sort no melhor caso ele rodou em 0,00021 de segundos e no pior
caso em 0,00135. O merge sort no melhor caso ele rodou em 0,00097 de
segundos e no pior caso ele rodou em 0,01906 de segundos.

    Segundo o Professor Moacir Ponti Jr. do Instituto de Ciências
Matemáticas e de Computação da USP, o algoritmo bubble sort consome
muito tempo quando todos os itens estão ordenados.

    \subsubsection{5 - Análise Experimental
Opcional}\label{anuxe1lise-experimental-opcional}

Esta seção opcional deve conter: * A descrição da configuração da
análise experimental conduzida: implementação utilizada, configurações
da máquina utilizada, tamanhos de entrada utilizados, tipo de dados
usados como entrada, e método utilizado de quantificação de
tempo/operações primitivas. * Gráfico comparando ambas curvas de
desempenho obtidas pelos algoritmos de ordenação escolhidos.

    \paragraph{Valores de Test}\label{valores-de-test}

    \begin{Verbatim}[commandchars=\\\{\}]
{\color{incolor}In [{\color{incolor}18}]:} \PY{k+kn}{from} \PY{n+nn}{random} \PY{k+kn}{import} \PY{n}{shuffle}
         \PY{k+kn}{import} \PY{n+nn}{threading}
         \PY{k+kn}{import} \PY{n+nn}{time}
\end{Verbatim}


    \begin{Verbatim}[commandchars=\\\{\}]
{\color{incolor}In [{\color{incolor}19}]:} \PY{n}{x0} \PY{o}{=} \PY{n+nb}{range}\PY{p}{(}\PY{l+m+mi}{100}\PY{p}{)}     \PY{c+c1}{\PYZsh{} 0}
         \PY{n}{x1} \PY{o}{=} \PY{n+nb}{range}\PY{p}{(}\PY{l+m+mi}{500}\PY{p}{)}     \PY{c+c1}{\PYZsh{} 1}
         \PY{n}{x2} \PY{o}{=} \PY{n+nb}{range}\PY{p}{(}\PY{l+m+mi}{1000}\PY{p}{)}    \PY{c+c1}{\PYZsh{} 2}
         \PY{n}{x3} \PY{o}{=} \PY{n+nb}{range}\PY{p}{(}\PY{l+m+mi}{1500}\PY{p}{)}    \PY{c+c1}{\PYZsh{} 3 }
         \PY{n}{x4} \PY{o}{=} \PY{n+nb}{range}\PY{p}{(}\PY{l+m+mi}{2500}\PY{p}{)}   \PY{c+c1}{\PYZsh{} 4 }
         \PY{n}{x5} \PY{o}{=} \PY{n+nb}{range}\PY{p}{(}\PY{l+m+mi}{3000}\PY{p}{)}   \PY{c+c1}{\PYZsh{} 5}
         \PY{n}{x6} \PY{o}{=} \PY{n+nb}{range}\PY{p}{(}\PY{l+m+mi}{35000}\PY{p}{)}   \PY{c+c1}{\PYZsh{} 6}
         \PY{n}{x7} \PY{o}{=} \PY{n+nb}{range}\PY{p}{(}\PY{l+m+mi}{40000}\PY{p}{)}  \PY{c+c1}{\PYZsh{} 7 }
         \PY{n}{test} \PY{o}{=} \PY{p}{[}\PY{n}{x0}\PY{p}{,}\PY{n}{x1}\PY{p}{,}\PY{n}{x2}\PY{p}{,}\PY{n}{x3}\PY{p}{,}\PY{n}{x4}\PY{p}{,}\PY{n}{x5}\PY{p}{,}\PY{n}{x6}\PY{p}{,}\PY{n}{x7}\PY{p}{]}
\end{Verbatim}


    \begin{Verbatim}[commandchars=\\\{\}]
{\color{incolor}In [{\color{incolor}20}]:} \PY{k}{for} \PY{n}{p} \PY{o+ow}{in} \PY{n+nb}{range}\PY{p}{(}\PY{n+nb}{len}\PY{p}{(}\PY{n}{test}\PY{p}{)}\PY{p}{)}\PY{p}{:}
             \PY{n}{shuffle}\PY{p}{(}\PY{n}{test}\PY{p}{[}\PY{n}{p}\PY{p}{]}\PY{p}{)}
\end{Verbatim}


    \begin{Verbatim}[commandchars=\\\{\}]
{\color{incolor}In [{\color{incolor}21}]:} \PY{k}{def} \PY{n+nf}{Metrica\PYZus{}Bubble}\PY{p}{(}\PY{p}{)}\PY{p}{:}
          \PY{n}{data} \PY{o}{=} \PY{p}{[}\PY{p}{]}
          \PY{k}{for} \PY{n}{p} \PY{o+ow}{in} \PY{n+nb}{range}\PY{p}{(}\PY{n+nb}{len}\PY{p}{(}\PY{n}{test}\PY{p}{)}\PY{p}{)}\PY{p}{:}
             \PY{n}{ini} \PY{o}{=} \PY{n}{time}\PY{o}{.}\PY{n}{time}\PY{p}{(}\PY{p}{)}
             \PY{n}{bubble\PYZus{}sort}\PY{p}{(}\PY{n}{test}\PY{p}{[}\PY{n}{p}\PY{p}{]}\PY{p}{)}
             \PY{n}{fim} \PY{o}{=} \PY{n}{time}\PY{o}{.}\PY{n}{time}\PY{p}{(}\PY{p}{)}
             \PY{k}{print} \PY{l+s+s2}{\PYZdq{}}\PY{l+s+se}{\PYZbs{}n}\PY{l+s+s2}{\PYZdq{}}\PY{p}{,}\PY{n}{p}\PY{p}{,}\PY{l+s+s2}{\PYZdq{}}\PY{l+s+s2}{[Bubble] }\PY{l+s+se}{\PYZbs{}t}\PY{l+s+se}{\PYZbs{}n}\PY{l+s+s2}{ Tempo de Execução: }\PY{l+s+s2}{\PYZdq{}}\PY{p}{,}\PY{n}{fim}\PY{o}{\PYZhy{}}\PY{n}{ini}
             \PY{n}{tempo} \PY{o}{=} \PY{n}{fim}\PY{o}{\PYZhy{}}\PY{n}{ini} 
             \PY{n}{data}\PY{o}{.}\PY{n}{append}\PY{p}{(}\PY{n}{tempo}\PY{p}{)}
          \PY{k}{return} \PY{n}{data}  
         
         \PY{k}{def} \PY{n+nf}{Metrica\PYZus{}Merge}\PY{p}{(}\PY{p}{)}\PY{p}{:}
          \PY{n}{data} \PY{o}{=} \PY{p}{[}\PY{p}{]}
          \PY{k}{for} \PY{n}{p} \PY{o+ow}{in} \PY{n+nb}{range}\PY{p}{(}\PY{n+nb}{len}\PY{p}{(}\PY{n}{test}\PY{p}{)}\PY{p}{)}\PY{p}{:}
             \PY{n}{ini} \PY{o}{=} \PY{n}{time}\PY{o}{.}\PY{n}{time}\PY{p}{(}\PY{p}{)}
             \PY{n}{merge\PYZus{}sort}\PY{p}{(}\PY{n}{test}\PY{p}{[}\PY{n}{p}\PY{p}{]}\PY{p}{)}
             \PY{n}{fim} \PY{o}{=} \PY{n}{time}\PY{o}{.}\PY{n}{time}\PY{p}{(}\PY{p}{)}
             \PY{k}{print} \PY{l+s+s2}{\PYZdq{}}\PY{l+s+se}{\PYZbs{}n}\PY{l+s+s2}{\PYZdq{}}\PY{p}{,}\PY{n}{p}\PY{p}{,}\PY{l+s+s2}{\PYZdq{}}\PY{l+s+s2}{[Merge] }\PY{l+s+se}{\PYZbs{}t}\PY{l+s+se}{\PYZbs{}n}\PY{l+s+s2}{ Tempo de Execução: }\PY{l+s+s2}{\PYZdq{}}\PY{p}{,}\PY{n}{fim}\PY{o}{\PYZhy{}}\PY{n}{ini}
             \PY{n}{tempo} \PY{o}{=} \PY{n}{fim}\PY{o}{\PYZhy{}}\PY{n}{ini} 
             \PY{n}{data}\PY{o}{.}\PY{n}{append}\PY{p}{(}\PY{n}{tempo}\PY{p}{)}
          \PY{k}{return} \PY{n}{data}  
\end{Verbatim}


    \subsubsection{"Rodando Metricas com
Mult-thread"}\label{rodando-metricas-com-mult-thread}

    \begin{Verbatim}[commandchars=\\\{\}]
{\color{incolor}In [{\color{incolor}22}]:} \PY{n}{t} \PY{o}{=} \PY{n}{threading}\PY{o}{.}\PY{n}{Thread}\PY{p}{(}\PY{n}{target}\PY{o}{=}\PY{n}{Metrica\PYZus{}Bubble}\PY{p}{,}\PY{p}{)}
         \PY{n}{t2} \PY{o}{=} \PY{n}{threading}\PY{o}{.}\PY{n}{Thread}\PY{p}{(}\PY{n}{target}\PY{o}{=}\PY{n}{Metrica\PYZus{}Merge}\PY{p}{,}\PY{p}{)}
         \PY{n}{t}\PY{o}{.}\PY{n}{start}\PY{p}{(}\PY{p}{)}
         \PY{n}{t2}\PY{o}{.}\PY{n}{start}\PY{p}{(}\PY{p}{)}
\end{Verbatim}


    \begin{Verbatim}[commandchars=\\\{\}]

0 [Merge] 	
 Tempo de Execução:  0.00234603881836

1 [Merge] 	
 Tempo de Execução:  0.00824189186096

2 [Merge] 	
 Tempo de Execução:  0.0161800384521

0 [Bubble] 	
 Tempo de Execução:  0.045725107193

3 [Merge] 	
 Tempo de Execução:  0.0324440002441

4 [Merge] 	
 Tempo de Execução:  0.0396988391876

5 [Merge] 	
 Tempo de Execução:  0.0295140743256

1 [Bubble] 	
 Tempo de Execução:  0.126824140549

6 [Merge] 	
 Tempo de Execução:  0.540427923203

7 [Merge] 	
 Tempo de Execução:  0.719171047211

2 [Bubble] 	
 Tempo de Execução:  1.25040483475

3 [Bubble] 	
 Tempo de Execução:  0.215059995651

4 [Bubble] 	
 Tempo de Execução:  0.574501991272

5 [Bubble] 	
 Tempo de Execução:  0.72434592247

6 [Bubble] 	
 Tempo de Execução:  96.4967339039

7 [Bubble] 	
 Tempo de Execução:  126.438951015

    \end{Verbatim}

    \subsubsection{Metricas sem Mult-thread}\label{metricas-sem-mult-thread}

    \begin{Verbatim}[commandchars=\\\{\}]
{\color{incolor}In [{\color{incolor}12}]:} \PY{n}{Metrica\PYZus{}bubble} \PY{o}{=} \PY{n}{Metrica\PYZus{}Bubble}\PY{p}{(}\PY{p}{)}
         \PY{n}{Metrica\PYZus{}merge} \PY{o}{=} \PY{n}{Metrica\PYZus{}Merge}\PY{p}{(}\PY{p}{)}
\end{Verbatim}


    \begin{Verbatim}[commandchars=\\\{\}]

0 [Bubble] 	
 Tempo de Execução:  0.000216007232666

1 [Bubble] 	
 Tempo de Execução:  0.000945091247559

2 [Bubble] 	
 Tempo de Execução:  0.000577211380005

3 [Bubble] 	
 Tempo de Execução:  0.000468015670776

4 [Bubble] 	
 Tempo de Execução:  0.000679016113281

5 [Bubble] 	
 Tempo de Execução:  0.000565052032471

6 [Bubble] 	
 Tempo de Execução:  0.000970840454102

7 [Bubble] 	
 Tempo de Execução:  0.00135803222656

0 [Merge] 	
 Tempo de Execução:  0.000979900360107

1 [Merge] 	
 Tempo de Execução:  0.00390791893005

2 [Merge] 	
 Tempo de Execução:  0.0108478069305

3 [Merge] 	
 Tempo de Execução:  0.013839006424

4 [Merge] 	
 Tempo de Execução:  0.0141232013702

5 [Merge] 	
 Tempo de Execução:  0.0141110420227

6 [Merge] 	
 Tempo de Execução:  0.0157148838043

7 [Merge] 	
 Tempo de Execução:  0.0190689563751

    \end{Verbatim}

    \subsubsection{Plot Grafico de Desempenho Merge
Sort}\label{plot-grafico-de-desempenho-merge-sort}

    \begin{Verbatim}[commandchars=\\\{\}]
{\color{incolor}In [{\color{incolor}13}]:} \PY{k+kn}{import} \PY{n+nn}{matplotlib.pyplot} \PY{k+kn}{as} \PY{n+nn}{plt}
         \PY{k+kn}{import} \PY{n+nn}{seaborn} \PY{k+kn}{as} \PY{n+nn}{sns}
         \PY{o}{\PYZpc{}}\PY{k}{matplotlib} inline
\end{Verbatim}


    \begin{Verbatim}[commandchars=\\\{\}]
{\color{incolor}In [{\color{incolor}14}]:} \PY{n}{plt}\PY{o}{.}\PY{n}{plot}\PY{p}{(}\PY{n}{Metrica\PYZus{}merge}\PY{p}{)}
         \PY{n}{plt}\PY{o}{.}\PY{n}{ylabel}\PY{p}{(}\PY{l+s+s1}{\PYZsq{}}\PY{l+s+s1}{Merge Sort}\PY{l+s+s1}{\PYZsq{}}\PY{p}{)}
         \PY{n}{plt}\PY{o}{.}\PY{n}{show}\PY{p}{(}\PY{p}{)}
\end{Verbatim}


    \begin{center}
    \adjustimage{max size={0.9\linewidth}{0.9\paperheight}}{output_42_0.png}
    \end{center}
    { \hspace*{\fill} \\}
    
    \subsubsection{Plot Grafico de Desempenho bubble
sort}\label{plot-grafico-de-desempenho-bubble-sort}

    \begin{Verbatim}[commandchars=\\\{\}]
{\color{incolor}In [{\color{incolor}15}]:} \PY{n}{plt}\PY{o}{.}\PY{n}{plot}\PY{p}{(}\PY{n}{Metrica\PYZus{}bubble}\PY{p}{)}
         \PY{n}{plt}\PY{o}{.}\PY{n}{ylabel}\PY{p}{(}\PY{l+s+s1}{\PYZsq{}}\PY{l+s+s1}{bubble sort}\PY{l+s+s1}{\PYZsq{}}\PY{p}{)}
         \PY{n}{plt}\PY{o}{.}\PY{n}{show}\PY{p}{(}\PY{p}{)}
\end{Verbatim}


    \begin{center}
    \adjustimage{max size={0.9\linewidth}{0.9\paperheight}}{output_44_0.png}
    \end{center}
    { \hspace*{\fill} \\}
    
    \subsubsection{6 - Referências}\label{referuxeancias}

Esta seção deve conter uma linha para cada referência utilizada.
Exemplo:

Lee, J., \& Yeung, C. Y. (2018). Personalizing Lexical Simplification.
In Proceedings of the 27th International Conference on Computational
Linguistics. Mancini, P. (2011). Leader, president, person: Lexical
ambiguities and interpretive implications. European Journal of
Communication, 26(1). Saggion, H. (2018). LaSTUS/TALN at Complex Word
Identification (CWI) 2018 Shared Task. In Proceedings of the Thirteenth
Workshop on Innovative Use of NLP for Building Educational Applications

MERGESORT -
https://www.ft.unicamp.br/liag/siteEd/includes/arquivos/MergeSortResumo\_Grupo4\_ST364A\_2010.pdf

Merge Sort -
https://pt.slideshare.net/dianacarolinatarapueschirivi/merge-sort-25398213

bubble sort -
http://www2.dcc.ufmg.br/disciplinas/aeds2\_turmaA1/bubblesort.pdf

https://edisciplinas.ups.br/pluginfile.php/2223654/mod\_resource/content/1/ICC2\_ordenacao\_parte1.pdf

https://www.ufrj.brb/jairo\_souza/files/2009/12/2-Ordena\%C3\%A7\%C3\%A3o-BubbleSort.pdf


    % Add a bibliography block to the postdoc
    
    
    
    \end{document}
